\documentclass{article}
\usepackage[fancyhdr,pdf]{latex2man}

\input{common.tex}

\begin{document}

\begin{Name}{3}{libunwind-nto}{Blackberry}{Programming Library}{QNX Neutrino support in libunwind}libunwind-nto -- QNX Neutrino support in libunwind
\end{Name}

\section{Synopsis}

\File{\#include $<$libunwind-nto.h$>$}\\

\noindent
\Type{unw\_accessors\_t} \Var{unw\_nto\_accessors};\\

\Type{void~*}\Func{unw\_nto\_create}(\Type{pid\_t}, \Type{pthread\_t});\\
\noindent
\Type{void} \Func{unw\_nto\_destroy}(\Type{void~*});\\

\noindent
\Type{int} \Func{unw\_nto\_find\_proc\_info}(\Type{unw\_addr\_space\_t}, \Type{unw\_word\_t}, \Type{unw\_proc\_info\_t~*}, \Type{int}, \Type{void~*});\\
\noindent
\Type{void} \Func{unw\_nto\_put\_unwind\_info}(\Type{unw\_addr\_space\_t}, \Type{unw\_proc\_info\_t~*}, \Type{void~*});\\
\noindent
\Type{int} \Func{unw\_nto\_get\_dyn\_info\_list\_addr}(\Type{unw\_addr\_space\_t}, \Type{unw\_word\_t~*}, \Type{void~*});\\
\noindent
\Type{int} \Func{unw\_nto\_access\_mem}(\Type{unw\_addr\_space\_t}, \Type{unw\_word\_t}, \Type{unw\_word\_t~*}, \Type{int}, \Type{void~*});\\
\noindent
\Type{int} \Func{unw\_nto\_access\_reg}(\Type{unw\_addr\_space\_t}, \Type{unw\_regnum\_t}, \Type{unw\_word\_t~*}, \Type{int}, \Type{void~*});\\
\noindent
\Type{int} \Func{unw\_nto\_access\_fpreg}(\Type{unw\_addr\_space\_t}, \Type{unw\_regnum\_t}, \Type{unw\_fpreg\_t~*}, \Type{int}, \Type{void~*});\\
\noindent
\Type{int} \Func{unw\_nto\_get\_proc\_name}(\Type{unw\_addr\_space\_t}, \Type{unw\_word\_t}, \Type{char~*}, \Type{size\_t}, \Type{unw\_word\_t~*}, \Type{void~*});\\
\noindent
\Type{int} \Func{unw\_nto\_resume}(\Type{unw\_addr\_space\_t}, \Type{unw\_cursor\_t~*}, \Type{void~*});\\

\section{Description}

The QNX operating system makes it possible for a process to
gain access to the machine-state and virtual memory of \emph{another}
process.
With the right set of call-back routines,
it is therefore possible to hook up \Prog{libunwind} to another process.
While it's not very difficult to do so directly,
\Prog{libunwind} further facilitates this task by providing
ready-to-use callbacks for this purpose.
The routines and variables
implementing this facility use a name-prefix of \Func{unw\_nto},
which is stands for ``unwind-via-nto''.

An application that wants to use the \Func{unw\_nto}-facility first needs
to create a new \Prog{libunwind} address-space that represents the
target process.  This is done by calling
\Func{unw\_create\_addr\_space}().  In many cases, the application
will simply want to pass the address of \Var{unw\_nto\_accessors} as the
first argument to this routine.  Doing so will ensure that
\Prog{libunwind} will be able to properly unwind the target process.
However, in special circumstances, an application may prefer to use
only portions of the \Prog{unw\_nto}-facility.  For this reason, the
individual callback routines (\Func{unw\_nto\_find\_proc\_info}(),
\Func{unw\_nto\_put\_unwind\_info}(), etc.)  are also available for direct
use.  Of course, the addresses of these routines could also be picked
up from \Var{unw\_nto\_accessors}, but doing so would prevent static
initialization.  Also, when using \Var{unw\_nto\_accessors}, \emph{all}
the callback routines will be linked into the application, even if
they are never actually called.

The process-ID (pid) of the target process must be known,
and only a single thread can be unwound at a time so the thread-id (tid)
must also be specified.
A \Prog{unw\_nto}-info-structure can be created
by calling \Func{unw\_nto\_create}(), passing the pid and tid of the target
process thread as the arguments.
This will stop the target thread.
The returned void-pointer then needs to be
passed as the ``argument'' pointer (third argument) to
\Func{unw\_init\_remote}().

The \Func{unw\_nto\_resume}() routine can be used to resume execution of
the target process after unwinding.

When the application is done using \Prog{libunwind} on the target
process, \Func{unw\_nto\_destroy}() needs to be called, passing it the
void-pointer that was returned by the corresponding call to
\Func{unw\_nto\_create}().  This ensures that all memory and other
resources are freed up.

\section{Thread Safety}

The \Prog{unw\_nto}-facility assumes that a single \Prog{unw\_nto}-info
structure is never shared between threads of the unwinding program.
Because of this,
no explicit locking is used.
As long as only one thread uses a \Prog{unw\_nto}-info structure at any given time,
this facility is thread-safe.

\section{Return Value}

\Func{unw\_nto\_create}() may return a \Const{NULL} pointer if it fails
to create the \Prog{unw\_nto}-info-structure for any reason.

\section{Files}

\begin{Description}
\item[\File{libunwind-nto.h}] Headerfile to include when using the
  interface defined by this library.
\item[\Opt{-l}\File{unwind-nto} \Opt{-l}\File{unwind-generic}]
    Linker-switches to add when building a program that uses the
    functions defined by this library.
\end{Description}

\section{See Also}

\SeeAlso{libunwind(3)},

\LatexManEnd

\end{document}
