\documentclass{article}
\usepackage[fancyhdr,pdf]{latex2man}

\input{common.tex}

\begin{document}

\begin{Name}{3}{unw\_resume}{David Mosberger-Tang}{Programming Library}{unw\_resume}

  unw\_resume -- resume execution in a particular stack frame
\end{Name}

\section{Synopsis}

\File{\#include $<$libunwind.h$>$}\\

\Type{int} \Func{unw\_resume}(\Type{unw\_cursor\_t~*}\Var{cursor});\\

\section{Description}

The \Func{unw\_resume}() routine resumes execution at the stack frame
identified by \Var{cursor}.  Normally, this is accomplished by
restoring the ``preserved'' (callee-saved) machine state.  However, if
execution in any of the stack frames younger (more deeply nested) than
the one identified by \Var{cursor} was interrupted by a signal, then
\Func{unw\_resume}() will restore the entire machine state, including
the ``preserved'' and ``scratch'' (caller-saved) registers, as well as
the signal mask.

Most platforms reserve some registers to pass arguments to exception
handlers (e.g., IA-64 uses \texttt{r15}-\texttt{r18} for this
purpose).  These registers are normally treated like ``scratch''
registers.  However, if \Prog{libunwind} is used to define an
exception argument register, e.g., by calling \Func{unw\_set\_reg}(),
then \Func{unw\_resume}() will always install the new value as the
contents of that register.  In other words, the exception handling
arguments are installed even in cases where normally only the
``preserved'' registers are restored.

Note that \Func{unw\_resume}() does \emph{not} invoke any unwind
handlers (aka, ``personality routines'').  If a program needs this, it
will have to do so on its own by obtaining the \Type{unw\_proc\_info\_t}
of each unwound frame and appropriately processing its unwind handler
and language-specific data area (lsda).  These steps are generally
dependent on the target-platform and are regulated by the
processor-specific ABI (application-binary interface).

\section{Return Value}

For local unwinding, \Func{unw\_resume}() does not return on success.
For remote unwinding, it returns 0 on success.  On failure, the
negative value of one of the errors below is returned.

\section{Errors}

\begin{Description}
\item[\Const{UNW\_EUNSPEC}] An unspecified error occurred.
\item[\Const{UNW\_EBADREG}] A register needed by \Func{unw\_resume}() wasn't
  accessible.
\item[\Const{UNW\_EINVALIDIP}] The instruction pointer identified by
  \Var{cursor} is not valid.
\item[\Const{UNW\_BADFRAME}] The stack frame identified by
  \Var{cursor} is not valid.
\end{Description}

\section{See Also}

\SeeAlso{libunwind(3)},
\SeeAlso{unw\_set\_reg(3)},
sigprocmask(2)

\section{Author}

\noindent
David Mosberger-Tang\\
Hewlett-Packard Labs\\
Palo-Alto, CA 94304\\
Email: \Email{davidm@hpl.hp.com}\\
WWW: \URL{http://www.hpl.hp.com/research/linux/libunwind/}.
\LatexManEnd

\end{document}
